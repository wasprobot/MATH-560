\documentclass[boxes, qed]{homework}
\usepackage{xcolor}
\usepackage{amsmath}
\usepackage{listings}
\name{Rohit Wason}
\course{Math 560}
\term{Spring 2021}
\hwnum{(\#2)}

\newcommand{\bigzero}{\mbox{\normalfont\Large\bfseries 0}}
\newcommand{\rvline}{\hspace*{-\arraycolsep}\vline\hspace*{-\arraycolsep}}

\begin{document}

\newenvironment{amatrix}[1]{%
  \left[\begin{array}{@{}*{#1}{c}|c@{}}
}{%
  \end{array}\right]
}

\newenvironment{augmatrix}[1]{%
  \left[\begin{array}{#1}
}{%
  \end{array}\right]
}

\begin{problem}Let $x_i=(i - 21)/10$ for $i = 1,\dots, 41$.
  This creates an equally-spaced grid of values between
  $-2$ and $2$ with increments of $1/10$.
\end{problem}
\begin{solution}
  (a) Write an R command to create a vector $x$ with 
  elements $x_1, . . . , x_{41}$.
  \begin{lstlisting}[backgroundcolor = \color{lightgray},language = R]
    x <- ((1:41)-21)/10
  \end{lstlisting}

  (b) Let $u_i = \frac{1}{\sqrt{2\pi}}e^{-x_i/2}$
  for $i = 1, . . . , 41$. Write an R command to create a vector u with elements
  $u_1, . . . u_{41}$.
  \begin{lstlisting}[backgroundcolor = \color{lightgray},language = R]
    u <- exp(-(1:41)/2)/sqrt(2*pi)
  \end{lstlisting}

  (c) Write an R command to give the elements of $x$
  corresponding to values of $u_i$ which are greater
  than $\frac{1}{4}$.
  \begin{lstlisting}[backgroundcolor = \color{lightgray},language = R]
    x[u > 0.25]
  \end{lstlisting}
  PS: There no values in $u$ that are greater than $\frac{1}{4}$.
\end{solution}
\begin{problem} Type the following 3 command in R:
  \begin{lstlisting}[backgroundcolor = \color{lightgray},language = R]
    initials=c("GZ","VA","TK","BH","LM","EY")
    quiz.grades=c(28,15,21,30,24,10)
    exam.grades=c(86,72,50,97,90,55)
  \end{lstlisting}
  The $i^{th}$ component of each vector gives the value of the respective 
  variable for the $i^{th}$ person in a class.
\end{problem}
\begin{solution}
  (a) The maximum possible number of points on the quiz is $30$. Write an R command to convert
  the vector of quiz grades to a vector of percentages. 
  (For example, a quiz grade of 27 should be converted to $90\%$.)
  \begin{lstlisting}[backgroundcolor = \color{lightgray},language = R]
    quiz.grades*(100/30)
  \end{lstlisting}
  (b) Students who scored at least $18$ passed the quiz. 
  Students who scored at least $60$ passed the
  exam. Write an R command to give the initials of 
  all students who both passed the quiz and passed the
  exam.
  \begin{lstlisting}[backgroundcolor = \color{lightgray},language = R]
    initials[quiz.grades>=18 & exam.grades>=60]
  \end{lstlisting}
  (c) Write an R command to add a 3 point curve to the vector of exam grades.
  \begin{lstlisting}[backgroundcolor = \color{lightgray},language = R]
    exam.grades = exam.grades + 3
  \end{lstlisting}
\end{solution}
\begin{problem}Consider two tosses of a fair coin.
\end{problem}
\begin{solution}
  The possible events are:
  \begin{tabular}{|l|l|l|l|}
    \hline
    HH & HT & TH & TT \\
    \hline
  \end{tabular}\\

  (a) Let $X$ be the number of heads minus the number of tails in the two tosses.
  The probability distribution of $X$ is:\\

  \begin{tabular}{|l|l|l|l|}
    \hline
    \textbf{Value (X)} & 2 & 0 & -2 \\
    \hline
    \textbf{Probability} & 0.25 & 0.5 & 0.25 \\
    \hline
  \end{tabular}\\

  (b)  Let $Y$ be the number of heads before the 
  first time tails occurs. (If tails does not occur in the
  two tosses, then $Y = 2$).
  The probability distribution of $Y$ is:\\

  \begin{tabular}{|l|l|l|l|}
    \hline
    \textbf{Value (Y)} & 2 & 1 & 0 \\
    \hline
    \textbf{Probability} & 0.25 & 0.25 & 0.5 \\
    \hline
  \end{tabular}
\end{solution}
\end{document}
