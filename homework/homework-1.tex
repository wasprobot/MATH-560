\documentclass[boxes, qed]{homework}
\usepackage{amsmath}
\name{Rohit Wason}
\course{Math 560}
\term{Spring 2021}
\hwnum{(\#1)}

\newcommand{\bigzero}{\mbox{\normalfont\Large\bfseries 0}}
\newcommand{\rvline}{\hspace*{-\arraycolsep}\vline\hspace*{-\arraycolsep}}

\begin{document}

\newenvironment{amatrix}[1]{%
  \left[\begin{array}{@{}*{#1}{c}|c@{}}
}{%
  \end{array}\right]
}

\newenvironment{augmatrix}[1]{%
  \left[\begin{array}{#1}
}{%
  \end{array}\right]
}

\problemnumber{1}
\begin{problem}
  Canada has two official languages (French and English). 
  The distribution of responses to the question: “What is your mother tongue?”
  is provided.
\end{problem}
\begin{solution}
  a) What probability should replace “?” in the distribution?\\
  
  Let the probability that a Canadian speaks English be represented by $P(E)=0.59$, 
  French by $P(F)=?$, Asian/Pacific by $P(A)=0.07$ and Other by $P(O)=0.11$.\\

  Since these are exhaustive (they complete the selection of languages spoken 
  in Canada), $P(E)+P(F)+P(A)+P(O) = 1$. Therefore we have:
  \begin{align*}
    P(F) &= 1-(P(E)+P(A)+P(O))\\
    &= 1-(0.59+0.07+0.11)\\
    & = 1-0.77\\
    & = \boxed{0.23}
  \end{align*}

  b) What is the probability that the mother tongue of a randomly 
  selected Canadian is not English?\\

  Let's call $P(E^{'})$ the probability that the mother tongue
  is \textbf{not} English. Again, since the given probabilities 
  are exclusive and exhaustive, the possibilities that a randomly
  selected Canadian speaks a language other than English are same as
  the possibilities that they speak French \textbf{or} that they
  speak Asian/Pacific \textbf{or} they speak "Other". In other words:
  \begin{align*}
    P(E^{'}) &= P(F) + P(A) + P(O)\\
    &= 0.23 + 0.07 + 0.11\\
    &= \boxed{0.41}
  \end{align*}
\end{solution}
\begin{problem}
  Suppose that $45\%$ of adults in a study eat enough vegetables, 
  $40\%$ eat enough fruit, and
  $25\%$ do both.
\end{problem}
\begin{solution}
  a) What is the probability that a randomly selected adult 
  from this study eats enough vegetables or eats enough fruit?\\

  Let $P(V) = 0.45$ be the probability that a randomly selected
  adult eats enough vegetables; $P(F) = 0.40$ that they eat enough
  fruit; and $P(V \cap F) = 0.25$ that they do both.\\

  Let $P(V \cup {F})$ be the probability that a person eats
  enough vegetables \textbf{or} enough fruit. This probability
  would just be $P(V) + P(F)$ had these events been mutually exclusive.
  But since $P(V \cap F) \ne 0$, we need to discount $P(V \cap F)$
  so we don't count it again (this is also called \textbf{\textit{de Morgan's law}}):
  \begin{equation}
    P(A \cup B) = P(A) + P(B) - P(A \cap B)
  \end{equation}
  \begin{align*}
    \therefore P(V \cup F) &= P(V) + P(F) - P(V \cap F)\\
    &= 0.45 + 0.40 - 0.25\\
    &= \boxed{0.60}
  \end{align*}

  b) If 3 adults are randomly selected from this study 
  (independently of each other), what is the probability that
  at least one of them eats enough vegetables?\\

  Starting in the opposite direction,
  we can see that the probability that a randomly selected
  person does \textbf{not} eat enough vegetables,
  $$P(V{'}) = 1 - P(V) = 0.55$$

  Since the events of selecting the 3 persons are independent
  of each other (selecting one person doesn't affect which 
  subsequent person gets picked), the probability that
  \textbf{none of the 3 persons eat enough vegetables} is given by
  $$P(V{'})^3 = 0.55^3$$

  Therefore, the required probability that \textbf{at lease one} of these
  persons eats enough vegetables is the compliment of what we found above:
  $$=(1-0.55^3) = 0.8336$$
\end{solution}
\end{document}
