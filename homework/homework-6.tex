\documentclass[boxes, qed]{homework}
\usepackage{xcolor}
\usepackage{amsmath}
\usepackage{listings}
\name{Rohit Wason}
\course{Math 560}
\term{Spring 2021}
\hwnum{(\#6, Test of significance)}

\newcommand{\bigzero}{\mbox{\normalfont\Large\bfseries 0}}
\newcommand{\rvline}{\hspace*{-\arraycolsep}\vline\hspace*{-\arraycolsep}}

\begin{document}

\newenvironment{amatrix}[1]{%
  \left[\begin{array}{@{}*{#1}{c}|c@{}}
}{%
  \end{array}\right]
}

\newenvironment{augmatrix}[1]{%
  \left[\begin{array}{#1}
}{%
  \end{array}\right]
}

\begin{problem}To assess the accuracy of a laboratory scale, a standard weight is repeatedly 
  weighed a total of $n$ times. The scale readings are independent and Normally distributed 
  with an unknown mean $\mu$ and a known standard deviation $\sigma = 0.01$ grams.
\end{problem}
\begin{solution}a) How large does $n$ need to be to guarantee that a two-sided 95\% 
  confidence interval for $\mu$ has a margin of error no larger than $0.003$?\\

  We know that
  $$n\ge(\frac{z^*\sigma}{m})^2$$
  Also, for $C=95\%$, $z^*=1.96$.
  So we have that
  \begin{align*}
    n &\ge (\frac{1.96\times{0.01}}{0.003})^2\\
    n &\ge 42.69
  \end{align*}
  Thus we see that $n$ needs to be \textbf{at least} $\boxed{43}$.\\
  
  (b) What is the smallest confidence level for which a two-sided confidence interval for $\mu$ 
  is guaranteed to have margin of error no more than $0.003$ based on a sample size $n = 20$?\\
  
  Again, using the inequality above, 
  \begin{align*}
    20 &\ge (\frac{0.01z^*}{0.003})^2
    = 11.11{z^*}^2\\
    {z^*}^2 &\le 1.8\\
    \therefore z^* &\le 1.341
  \end{align*}

  For $z^*<0$ we are to use the area to the left of this point on the Normal curve.
  On looking up $z^*=1.341$ in the \textit{Normal table}, we find P-Value $=90.99\%$
  which is the desired confidence level.
\end{solution}

\begin{problem} A population follows a Normal distribution with mean $\mu$ and standard deviation $\sigma=3$.
  A random sample with replacement of size $n=10$ is taken from this population to test the hypothesis
  $H_0: \mu=5$.
\end{problem}
\begin{solution}(a) If the test of $H_0$ against the alternative $H_a: \mu<5$ and we observe the sample mean
  to be $\bar{x}=2$, should $H_0$ be rejected at level $\alpha=.05$?\\

  We are to compare $H_0: \mu=5$ vs. $H_a: \mu<5$. The statistic
  $$z=\frac{\bar{x}-\mu_o}{\sigma/\sqrt{n}} 
  = \frac{2-5}{3/\sqrt{10}}
  \approx -3.16 $$

  Looking up $-3.16$ in the \textit{Normal table}, we get the P-Value$=0.0008$
  which is significantly smaller than $\alpha=.05$, so $H_0$ is to be \textbf{rejected}.\\

  (b) If the test of $H_0$ against the alternative $H_a: \mu>5$ and we observe the sample mean to be
  $\bar{x}=2$, should $H_0$ be rejected at level $\alpha=.05$?\\

  From (a), since $z=-3.16$ the P-Value in this case will be the area to the \textit{right} of the
  probability at $z$, which $=1-P(z)=1-0.0008=0.9992$
  which is significantly larger than $\alpha=.05$, so $H_0$ is \textbf{not to be rejected}.\\
\end{solution}

\begin{problem}Your friend generated $n=5$ observations from a Normal distribution using a random number
  generator in R. Your friend remembers that the standard deviation was $\sigma=2$ but forgets what was used
  for the mean $\mu=?$.\\
  Your friend thinks that the value used for $\mu$ might have been $\mu_0=10$. 
  Is there strong evidence against this hypothesis? To answer this question, perform a test of significance 
  at level $\alpha=.05$. Carefully state the hypotheses, calculate an appropriate test statistic, 
  compute the P-value, and state your conclusion.
\end{problem}
\begin{solution}
  So the hypothesis $H_0$ states that the mean is $\mu_0=10$
  We observe, from the sample, that the sample mean, $\bar{x}=8.32287$.\\
  
  Hence we are to compute the test statistic, $z$ that represents
  the likelihood of $\mu$ being close to $\mu_0$. This statistic
  is given by:
  \begin{align*}
    z&=\frac{\bar{x}-\mu_o}{\sigma/\sqrt{n}}\\
    &= \frac{8.3228-10}{2/\sqrt{5}}\\
    &\approx -1.8751
  \end{align*}

  To find the P-Value for this statistic, we lookup $-1.8751$
  in the \textit{Normal table}, and since this value is to the left of $0$,
  the Normal mean, we're looking for the area to the left of $-1.8751$.\\

  P-Value $\approx 0.0307$. This value is smaller than our confidence-level
  $\alpha=0.05$, hence our friend's hypothesis $\mu_0=10$ must be \textbf{rejected}!
\end{solution}
\end{document}