\documentclass[boxes, qed]{homework}
\usepackage{xcolor,amsmath,listings}
\usepackage{graphicx}
\graphicspath{ {./images/} }

\name{Rohit Wason}
\course{Math 560}
\term{Spring 2021}
\hwnum{(\#7, Inference of Proportions)}

\newcommand{\bigzero}{\mbox{\normalfont\Large\bfseries 0}}
\newcommand{\rvline}{\hspace*{-\arraycolsep}\vline\hspace*{-\arraycolsep}}

\begin{document}

\begin{problem}
  A table of two-variables is given with:\\
  Sample ratio for the treatment group, $\hat{p_1}=\frac{82}{200745} \approx 0.0004085$\\
  Sample ratio for the control group, $\hat{p_2}=\frac{162}{201229} \approx 0.0008050$\\
  Pooled ratio, $\hat{p}=\frac{162+82}{200745+201229} \approx 0.0006070$
\end{problem}
\begin{solution}
  Hypothesis:\\
  $H_0$: There is no association between the row \& the column variables vs. \\
  $H_a$: There is an association between them. \\

  Test statistic, 
  \begin{align*}
    \chi^2=\sum{\frac{(\mathrm{observed count}-\mathrm{expected count})^2}{\mathrm{expected count}}}
  \end{align*}
  The correspnding P-Value for the two-sided test
  $=2\times pnorm(-5.10328) \approx 0.0000003$, 
  which is considerably smaller  than $\alpha=0.5$ so we \textbf{reject the null-hypothesis}.
\end{solution}
\end{document}