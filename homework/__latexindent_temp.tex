\documentclass[boxes, qed]{homework}
\usepackage{xcolor}
\usepackage{amsmath}
\usepackage{listings}
\name{Rohit Wason}
\course{Math 560}
\term{Spring 2021}
\hwnum{(\#3)}

\newcommand{\bigzero}{\mbox{\normalfont\Large\bfseries 0}}
\newcommand{\rvline}{\hspace*{-\arraycolsep}\vline\hspace*{-\arraycolsep}}

\begin{document}

\newenvironment{amatrix}[1]{%
  \left[\begin{array}{@{}*{#1}{c}|c@{}}
}{%
  \end{array}\right]
}

\newenvironment{augmatrix}[1]{%
  \left[\begin{array}{#1}
}{%
  \end{array}\right]
}

\begin{problem}Using the given probability distribution of $X$.
\end{problem}
\begin{solution}
  (a) $P(2 < X < 3)$
  % \begin{lstlisting}[backgroundcolor = \color{lightgray},language = R]
  %   x <- ((1:41)-21)/10
  % \end{lstlisting}
    \begin{align*}
      &= \int_{2}^{3}{\frac{2}{x^3}}dx\\
      &= 2\int_{2}^{3}{x^{-3}}dx\\
      &= -\left[ x^{-2} \right]_{2}^{3}\\
      &= -\left[ 3^{-2} - 2^{-2} \right]\\
      &\approx \boxed{0.12975}
    \end{align*}
    
    (b) $P(X < 2 \lor X > 3)$ can be seen as the probability
    of all events for $X>1$, excluding $P(2 < X < 3)$ (from above)
    and hence $= P(X > 1) - P(2 < X < 3)$.
    Since $P(X > 1)=1$, being the probability of all exhaustive cases

    $$P(X < 2 \lor X > 3) = 1-0.12975 = \boxed{0.87025}$$\\

    (c) $P(X < 2)$
    \begin{align*}
      &= \int_{-\infty}^{2}{\frac{2}{x^3}}dx\\
      &= \int_{-\infty}^{1}{\frac{2}{x^3}}dx + \int_{1}^{2}{\frac{2}{x^3}}dx\\
      &= 0 -\left[ x^{-2} \right]_{1}^{2}\\
      &= \left[ 1-\frac{1}{\sqrt{2}} \right]\\
      &\approx \boxed{0.29289}
    \end{align*}
\end{solution}

\begin{problem}The scores on a particular exam follow a $Normal(74, 8)$ distribution.
  On ``standardizing'' this variable we get
  $Z=\frac{X-\mu}{\sigma}$.
\end{problem}
\begin{solution}(a) Therefore the probability that a randomly selected student 
  has an exam score less than $80 = P(X<80) = P(Z<\frac{80-74}{8})$\\
  $=P(Z<0.75) = \boxed{0.7734}$ (using the \textit{Standard Normal Cumulative Probabilities}).
\end{solution}
\end{document}
