\documentclass[boxes, qed]{homework}
\usepackage{amsmath}
\name{Rohit Wason}
\course{Math 560}
\term{Spring 2021}
\hwnum{(\#1)}

\newcommand{\bigzero}{\mbox{\normalfont\Large\bfseries 0}}
\newcommand{\rvline}{\hspace*{-\arraycolsep}\vline\hspace*{-\arraycolsep}}

\begin{document}

\newenvironment{amatrix}[1]{%
  \left[\begin{array}{@{}*{#1}{c}|c@{}}
}{%
  \end{array}\right]
}

\newenvironment{augmatrix}[1]{%
  \left[\begin{array}{#1}
}{%
  \end{array}\right]
}

\problemnumber{1}
\begin{problem}
  Canada has two official languages (French and English). Here is the distribution of responses
  to the question: “What is your mother tongue?”\end{problem}
\begin{solution}
    \textbf{\textit{To prove:}} In other words we need to prove that the equation
    $3a + 5b = c$
    has positive solutions for all $c \ge 8$.\\

    \textbf{\textit{Basis step:}} Let $P(c)$ be the statement that the last statement is true for
    a given $c$. $P(8)$ holds true, since $a=1, b=1$ 
    are the positive solutions.\\

    \textbf{\textit{Induction step:}} Let's assume $P(k)$ holds for an arbitrary
    $k>8$. I.e., 
    $\exists a,b
    : a\ge{0},b\ge{0}
    : k=3a+5b$

    \begin{enumerate}
        \item[Case I] Both $a$ and $b$ are greater than zero.
        \begin{align*}
            \therefore k+1 &= 3a + 5b + 1 \\
            & = 3a + 5(b-1) + 6 \\
            & = 3(a+2) + 5(b-1)
        \end{align*}
        \item[Case II] $a=0$ and $b$ is greater than zero.
        \begin{align*}
            \therefore k+1 &= 5b + 1 \\
            & = 5(b-1) + 6 \\
            & = 3(2) + 5(b-1)
        \end{align*}
        \item[Case III] $a$ is greater than zero and $b=0$. 
        But this is only possible if
        $a\ge{3}$ since $k\ge{8}$.
        \begin{align*}
            \therefore k+1 &= 3a + 1 \\
            & = 3(a-3) + 1 \\
            & = 3(a-3) + 10\\
            & = 3(a-3) + 5(2)\\
        \end{align*}
    \end{enumerate}
    All the possible casese suggest that if $P(k)$ is true,
    $P(k+1)$ is true $\forall{k}\ge{8}\in{\mathbb{N}}$.
    I.e., any integer length $\ge{8}$ can be made out of
    these pipes.
\end{solution}
\end{document}
