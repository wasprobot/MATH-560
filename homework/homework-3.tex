\documentclass[boxes, qed]{homework}
\usepackage{xcolor}
\usepackage{amsmath}
\usepackage{listings}
\name{Rohit Wason}
\course{Math 560}
\term{Spring 2021}
\hwnum{(\#3)}

\newcommand{\bigzero}{\mbox{\normalfont\Large\bfseries 0}}
\newcommand{\rvline}{\hspace*{-\arraycolsep}\vline\hspace*{-\arraycolsep}}

\begin{document}

\newenvironment{amatrix}[1]{%
  \left[\begin{array}{@{}*{#1}{c}|c@{}}
}{%
  \end{array}\right]
}

\newenvironment{augmatrix}[1]{%
  \left[\begin{array}{#1}
}{%
  \end{array}\right]
}

\begin{problem}Using the given probability distribution of $X$.
\end{problem}
\begin{solution}(a) $P(2 < X < 3)$
  \begin{align*}
    &= \int_{2}^{3}{\frac{2}{x^3}}dx\\
    &= 2\int_{2}^{3}{x^{-3}}dx\\
    &= -\left[ x^{-2} \right]_{2}^{3}\\
    &= -\left[ 3^{-2} - 2^{-2} \right]\\
    &\approx \boxed{0.12975}
  \end{align*}

  (b) $P(X < 2 \lor X > 3)$ can be seen as the probability
    of all events for $X>1$, excluding $P(2 < X < 3)$ (from above)
    and hence $= P(X > 1) - P(2 < X < 3)$.
    Since $P(X > 1)=1$, being the probability of all exhaustive cases

    $$P(X < 2 \lor X > 3) = 1-0.12975 = \boxed{0.87025}$$\\

  (c) $P(X < 2)$
  \begin{align*}
    &= \int_{-\infty}^{2}{\frac{2}{x^3}}dx\\
    &= \int_{-\infty}^{1}{\frac{2}{x^3}}dx + \int_{1}^{2}{\frac{2}{x^3}}dx\\
    &= 0 -\left[ x^{-2} \right]_{1}^{2}\\
    &= \left[ 1-\frac{1}{\sqrt{2}} \right]\\
    &\approx \boxed{0.29289}
  \end{align*}
\end{solution}

\begin{problem}The scores on a particular exam follow a $Normal(74, 8)$ distribution.
  On ``standardizing'' this variable we get
  $Z=\frac{X-\mu}{\sigma}$.
\end{problem}
\begin{solution}(a) Therefore the probability that a randomly selected student 
  has an exam score less than $80 = P(X<80) = P(Z<\frac{80-74}{8})$\\
  $=P(Z<0.75) = \boxed{0.7734}$ (using the \textit{Standard Normal Cumulative Probabilities}).\\

  (b) The probability that a randomly selected student 
  has an exam score between 80 and 90
  $=P(80 < X < 90) = P(0.75 < Z < \frac{90-74}{8})$ like above.
  This is $=P(0.75 < Z < 2) = P(Z < 2) - P(Z < 0.75)$.\\
  Using the \textit{Standard Normal Cumulative Probabilities} we see that
  $P(Z < 0.75) = 0.7734$ and $P(Z < 2) = 0.9772$. Therefore
  $P(0.75 < Z < 2) = 0.9772 - 0.7734 = \boxed{0.2038}$\\

  (c) To find a value $x$ such that the probability that 
  a randomly selected student has an exam
  score less than $x$ is $0.90$ we reverse lookup
  the \textit{Standard Normal Cumulative Probabilities}
  for cummulative probablities close to $0.90$ and see that\\

  $P(Z<1.28) = 0.8997$ and $P(Z<1.29) = 0.9015$.
  From the definition, $X=Z\sigma + \mu$. Therefore
  $1.28(8)+74<x<1.29(8)+74$. Or
  $$84.24<x<84.32$$
\end{solution}

\begin{problem}The distributions of nonword- and word-errors are given. 
  The total number of words written are $500$. Also, the 
  the correlation between the number of nonword errors and the number 
  of word errors is $0.55$.\\
  Let $X$ be the number of word-, and $Y$
  the number of nonword-errors.
\end{problem}
\begin{solution}(a) The mean number of word-errors,
  $$\mu_X = \sum_{i=1}^{4}{x_ip_i}$$
  Using the given distribution, we have
  $$\mu_X = 0(0.2) + 1(0.4) + 2(0.3) + 3(0.1) = 1.3$$
  
  Similarly,
  $$\mu_Y = \sum_{i=1}^{3}{y_ip_i}$$
  since we only have 3 distinct values
  in the sequence. Again using the given distribution, we have
  $$\mu_Y = 0(0.6) + 1(0.3) + 2(0.1) = 0.5$$
  
  The mean of the total number of errors 
  (nonword errors plus word errors) is the sum
  of both the means: $\mu_{X+Y} = \mu_X + \mu_Y$
  $$=1.3+0.5 = \boxed{1.8}$$

  (b) To find the standard deviation, we first calculate
  the variance of each variable:
  \begin{align*}
    \sigma_{X}^2 &= \sum_{i=1}^4{(x_i-\mu_X)^2p_i}\\
    &= (0-1.3)^2(0.2)
      + (1-1.3)^2(0.4)
      + (2-1.3)^2(0.3)
      + (3-1.3)^2(0.1)\\
    &= 0.81
  \end{align*}
  Similarly
  \begin{align*}
    \sigma_{Y}^2 &= \sum_{i=1}^3{(y_i-\mu_Y)^2p_i}\\
    &= (0-0.5)^2(0.6)
      + (1-0.6)^2(0.3)
      + (2-0.6)^2(0.1)\\
    &= 0.394
  \end{align*}
  Now since $X$ and $Y$ have a correlation
  $\rho=0.55$ we have that the combined variance
  \begin{align*}
    \sigma_{X+Y}^2 
      &= \sigma_X^2 + \sigma_Y^2 + 2\rho\sigma_X\sigma_Y\\
      &= 0.81 + 0.394 + (0.55)\sqrt{0.81}\sqrt{0.394}\\
      &\boxed{\approx 1.5147}
  \end{align*}
\end{solution}
\begin{problem}Suppose that 70\% of homeowners in a town have a dog
  (let's call this event, $D$), 
  35\% have an alarm system (event $A$), and
  20\% have both a dog and an alarm system ($A\cap{D}$).
\end{problem}
\begin{solution}(a) According to \textit{de Morgan's Rule},
  the probability that a randomly selected homeowner 
  from this town has either a dog or an alarm system
  \begin{align*}
    P(A\cup{D})
      &= P(A) + P(D) - P(A\cap{D})\\
      &= 0.70 + 0.35 - 0.20\\
      &= \boxed{0.85}
  \end{align*}

  (b) The probability that a randomly selected homeowner 
  from this town has an alarm system given that the homeowner has a dog
  is given by
  \begin{align*}
    P(A|D) &= \frac{P(A\cap{D})}{P(D)}\\
    &= \frac{0.20}{0.70}\\
    &= 0.28
  \end{align*}
\end{solution}
\end{document}
