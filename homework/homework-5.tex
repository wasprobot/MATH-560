\documentclass[boxes, qed]{homework}
\usepackage{xcolor}
\usepackage{amsmath}
\usepackage{listings}
\name{Rohit Wason}
\course{Math 560}
\term{Spring 2021}
\hwnum{(\#5, Binomial)}

\newcommand{\bigzero}{\mbox{\normalfont\Large\bfseries 0}}
\newcommand{\rvline}{\hspace*{-\arraycolsep}\vline\hspace*{-\arraycolsep}}

\begin{document}

\newenvironment{amatrix}[1]{%
  \left[\begin{array}{@{}*{#1}{c}|c@{}}
}{%
  \end{array}\right]
}

\newenvironment{augmatrix}[1]{%
  \left[\begin{array}{#1}
}{%
  \end{array}\right]
}

\begin{problem}The probability distribution of a random variable
  $X$ is given, and $X_1$ and $X_2$ are random samples of size $2$.
  Assuming $X_1$ and $X_2$ are independent:
\end{problem}
\begin{solution}a) Since the variables are independent,
  the $9$ possible pairs $(X_1,X_2)$ and their probabilities are:
  \begin{enumerate}
    \item $P(0,0)=P(X_1=0 \land X_2=0)=P(X_1=0)P(X_2=0)=0.6^2=0.36$
    \item Similarly $P(0,1)=0.6*0.3=0.18$
    \item $P(0,2)=0.6*0.1=0.06$
    \item $P(1,0)=P(0,1)=0.18$
    \item $P(1,1)=0.3^2=0.09$
    \item $P(1,2)=0.3*0.1=0.03$
    \item $P(2,0)=P(0,2)=0.06$
    \item $P(2,1)=P(1,2)=0.03$, and
    \item $P(2,2)=0.1^2=0.01$\\
  \end{enumerate}

  b) As for the random variable $Y=\frac{X_1+X_2}{2}$
  the different possible values are:\\
  
  \begin{tabular}{|c|c|c|c|}
    \hline
    $Y$ & 0 & 1 & 2 \\
    \hline
    0 & 0 & 0.5 & 1 \\
    \hline
    1 & 0.5 & 1 & 1.5 \\
    \hline
    2 & 1 & 1.5 & 2 \\
    \hline
  \end{tabular}\\

  Therefore the distribution for $Y$, obtained by adding the
  $(X_1,X_2)$ probabilities from above, is:\\
  
  \begin{tabular}{|c|c|c|c|c|c|}
    \hline
    $Y$ & 0 & 0.5 & 1 & 1.5 & 2\\
    \hline
    $P(Y)$ & $0.36$ & $0.36$ & $0.21$ & $0.06$ & $0.01$\\
    \hline
  \end{tabular}
\end{solution}

\begin{problem}Population data on StatVillage (a hypothetical 128-block village in Canada) is given in
  the tab-delimited data file \texttt{StatVillage.txt}. The variables are listed in the first line of the data file, and
  information about the variables included in the file is given in the file \texttt{codesForStatVillage.txt}. 
  Use \texttt{R} or other computer software to answer the following questions:
  \begin{lstlisting}[backgroundcolor = \color{lightgray},language = R]
  population = read.table(file ="StatVillage.txt", 
    header = TRUE)
  \end{lstlisting}
\end{problem}
\begin{solution}(a) The variable labeled \texttt{TOTINCH} gives the total household income. Determine the proportion of
  households in this population with a total household income greater than $100,000$.
  \begin{lstlisting}[backgroundcolor = \color{lightgray},language = R]
  > nrow( population [ population$TOTINCH>100000 ,])
    /nrow( population )
  [1] 0.1113281
  \end{lstlisting}
  $\therefore \boxed{p\approx{0.1113}}$\\

  (b) If $100$ households are selected at random with replacement from this population, what is the
  probability that at least $10$ of the households in the sample will have a total household income greater
  than $100,000$? Compute the exact answer, rounded to at least 4 decimal places.\\

  Adding all probabilities from $X=10, 11, \dots, 100$ in $R$
  \begin{lstlisting}[backgroundcolor = \color{lightgray},language = R]
    prob=0; 
    for (n in c(10:100)) 
      prob = prob + dbinom(n, size=100, prob=0.1113); 
    prob;
  \end{lstlisting}
  we get $\boxed{P(X\ge{10}) = 0.6868}$\\

  (c) If $100$ households are selected at random with replacement from this population, then let $X$ be
  the number of households in the sample with income above $100,000$. What is the mean and the standard
  deviation of the sampling distribution of $X$?\\

  $X$ follows $\approx B(n,p) = B(100,0.11)$ from above.
  According to CLT, this also $\approx N(np,\sqrt{np(1-p)}) 
    =N(11.13,\sqrt{11.13(0.8887)})
    =N(11.13,3.1450)$\\

  Or $\boxed{\mu=11.13, \sigma=3.1450}$\\
  
  (d) Use the central limit theorem with a continuity correction to approximate the probability
  computed in part (b).\\
  
  We need to approximate $P(X\ge{10})$ which, with continuity correction
  really is $P(X\ge{9.5})$ since each bar in the histogram is centered
  on an integer and actually spans from the $\frac{1}{2}$
  or the preceeding integer.\\

  Using $\mu=11.13, \sigma=3.1450$ from above,
  for $Z=\frac{X-11.13}{3.1450}$, the required probability,
  \begin{align*}
    P(X\ge{9.5})
    &= P(\frac{X-11.13}{3.1450}\ge{\frac{9.5-11.13}{3.1450}})\\
    &= P(Z\ge\frac{-1.63}{3.1450}=-0.5183)\\
    &\approx 1-P(Z<-0.52)\\
    &= 1-0.3015\\
    &= \boxed{0.6985}
  \end{align*}
\end{solution}

\begin{problem}Your friend generated 5 observations from a normal distribution using a random number
  generator in R. Your friend remembers that the standard deviation was $\sigma=2$ but forgets 
  what was used for the mean $\mu$. Your friend generated and summarized the data using the following commands:
  \begin{lstlisting}[backgroundcolor = \color{lightgray},language = R]
  > set.seed(62341)
  > mu=####
  > sigma=2
  > x=rnorm(5,mean=mu,sd=sigma)
  > x
  [1] 8.742144 10.503939 7.301674 6.524349 8.542242
  > mean(x)
  [1] 8.32287
  > sd(x)
  [1] 1.521389
  \end{lstlisting}
\end{problem}
\begin{solution}
  Given:\\
  $n=5$\\
  $\mu_{\bar{X}}=8.32287$\\
  $\sigma_{\bar{X}}=1.521389$\\
  $\sigma=2$\\

  (a) The probability for producing the interval, $C=0.90$\\
  Looking up the Normal distribution table
  for $0.90+\frac{1-0.90}{2}=0.95$, we get
  $C=0.90 \implies z^*=1.645$\\
  Margin of error, $(z^*\sigma/\sqrt{n}) = 1.645(2)/\sqrt{5} = 1.4713$\\

  $\therefore$ the 90\% two-sided confidence interval for $\mu$, 
  $(\mu_{\bar{X}}\pm{1.4713})=\boxed{8.32287\pm{1.4713}}$\\

  (b) The probability for producing the interval, $C=0.995$\\
  Looking up the Normal distribution table
  for $0.995+\frac{1-0.995}{2}=0.9975$, we get
  $C=0.995 \implies z^*=2.81$\\
  Margin of error $= 2.81(2)/\sqrt{5} = 2.5133$\\

  $\therefore$ the 99.5\% two-sided confidence interval for $\mu$
  $=\boxed{8.32287\pm{2.5133}}$\\

  (c) By increasing the confidence level from $90\%$ to $99.5\%$, 
  we saw that the interval for estimating $\mu$ expanded. It seems
  that a 90\% confidence interval is a range of values that we can be 90\% 
  certain contains the true mean of the population. As the confidence level
  shrinks, we are less and less confident that the population mean
  could lie in the interval.\\

  It's easy to see that the two-sided 100\% confidence interval
  is the whole normal distribution ($\mu$ must lie somewhere on the graph!).
\end{solution}
\end{document}