\documentclass[boxes, qed]{homework}
\usepackage{xcolor}
\usepackage{amsmath}
\usepackage{listings}
\name{Rohit Wason}
\course{Math 560}
\term{Spring 2021}
\hwnum{(\#5, Binomial)}

\newcommand{\bigzero}{\mbox{\normalfont\Large\bfseries 0}}
\newcommand{\rvline}{\hspace*{-\arraycolsep}\vline\hspace*{-\arraycolsep}}

\begin{document}

\newenvironment{amatrix}[1]{%
  \left[\begin{array}{@{}*{#1}{c}|c@{}}
}{%
  \end{array}\right]
}

\newenvironment{augmatrix}[1]{%
  \left[\begin{array}{#1}
}{%
  \end{array}\right]
}

\begin{problem}
\end{problem}
\begin{solution}
\end{solution}

\begin{problem}Population data on StatVillage (a hypothetical 128-block village in Canada) is given in
  the tab-delimited data file \texttt{StatVillage.txt}. The variables are listed in the first line of the data file, and
  information about the variables included in the file is given in the file \texttt{codesForStatVillage.txt}. 
  Use \texttt{R} or other computer software to answer the following questions:
  \begin{lstlisting}[backgroundcolor = \color{lightgray},language = R]
    population = read.table(file ="StatVillage.txt", 
      header = TRUE)
  \end{lstlisting}
\end{problem}
\begin{solution}(a) The variable labeled \texttt{TOTINCH} gives the total household income. Determine the proportion of
  households in this population with a total household income greater than $100,000$.
  \begin{lstlisting}[backgroundcolor = \color{lightgray},language = R]
    nrow(population[population$TOTINCH>100000,])/
      nrow(population)
    \end{lstlisting}
    $\boxed{p=\frac{57}{512}\approx{0.11}}$\\

  (b) If $100$ households are selected at random with replacement from this population, what is the
  probability that \underline{at least $10$} of the households in the sample will have a total household income greater
  than $100,000$? Compute the exact answer, rounded to at least 4 decimal places.\\
  
  $X$, the random variable denoting the number of households among $n=100$, 
  whose income is greater than $100,000$ follows $\approx B(n,p) = B(100,0.11)$ from above.
  According to CLT, this also $\approx N(np,\sqrt{np(1-p)}) 
    =N(11,\sqrt{11(0.89)})
    =N(11,3.129)$\\

  The required probability, $P(X\ge{10})$ can be re-written as
  $$P(\frac{X-11}{3.129}\ge{\frac{10-11}{3.129}})$$
  For $Z=\frac{X-11}{3.129}$, this is
  \begin{align*}
    &= P(Z\ge\frac{-1}{3.129}=-0.32)\\
    &= 1-P(Z<-0.32)\\
    &= 1-0.3745\\
    &= \boxed{0.6255}
  \end{align*}
\end{solution}
\end{document}