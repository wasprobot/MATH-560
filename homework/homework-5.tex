\documentclass[boxes, qed]{homework}
\usepackage{xcolor}
\usepackage{amsmath}
\usepackage{listings}
\name{Rohit Wason}
\course{Math 560}
\term{Spring 2021}
\hwnum{(\#5, Binomial)}

\newcommand{\bigzero}{\mbox{\normalfont\Large\bfseries 0}}
\newcommand{\rvline}{\hspace*{-\arraycolsep}\vline\hspace*{-\arraycolsep}}

\begin{document}

\newenvironment{amatrix}[1]{%
  \left[\begin{array}{@{}*{#1}{c}|c@{}}
}{%
  \end{array}\right]
}

\newenvironment{augmatrix}[1]{%
  \left[\begin{array}{#1}
}{%
  \end{array}\right]
}

\begin{problem}
\end{problem}
\begin{solution}
\end{solution}

\begin{problem}Population data on StatVillage (a hypothetical 128-block village in Canada) is given in
  the tab-delimited data file \texttt{StatVillage.txt}. The variables are listed in the first line of the data file, and
  information about the variables included in the file is given in the file \texttt{codesForStatVillage.txt}. 
  Use \texttt{R} or other computer software to answer the following questions:
  \begin{lstlisting}[backgroundcolor = \color{lightgray},language = R]
    population = read.table(file ="StatVillage.txt", 
      header = TRUE)
  \end{lstlisting}
\end{problem}
\begin{solution}(a) The variable labeled \texttt{TOTINCH} gives the total household income. Determine the proportion of
  households in this population with a total household income greater than $100,000$.
  \begin{lstlisting}[backgroundcolor = \color{lightgray},language = R]
    nrow(population[population$TOTINCH>100000,])/
      nrow(population)
  \end{lstlisting}
  $\boxed{p=\frac{114}{1024}\approx{0.1113}}$\\

  (b) If $100$ households are selected at random with replacement from this population, what is the
  probability that at least $10$ of the households in the sample will have a total household income greater
  than $100,000$? Compute the exact answer, rounded to at least 4 decimal places.\\

  Adding all probabilities from $X=10, 11, \dots, 100$ in $R$
  \begin{lstlisting}[backgroundcolor = \color{lightgray},language = R]
    prob=0; 
    for (n in c(10:100)) 
      prob = prob + dbinom(n, size=100, prob=0.1113); 
    prob;
  \end{lstlisting}
  we get $\boxed{P(X\ge{10}) = 0.6868}$\\

  (c) If $100$ households are selected at random with replacement from this population, then let $X$ be
  the number of households in the sample with income above $100,000$. What is the mean and the standard
  deviation of the sampling distribution of $X$?\\

  $X$ follows $\approx B(n,p) = B(100,0.11)$ from above.
  According to CLT, this also $\approx N(np,\sqrt{np(1-p)}) 
    =N(11.13,\sqrt{11.13(0.8887)})
    =N(11.13,3.1450)$\\

  Or $\mu=11.13, \sigma=3.1450$\\
  
  (d) Use the central limit theorem with a continuity correction to approximate the probability
  computed in part (b)\\
  
  Using $\mu=11.13, \sigma=3.1450$ from (c),
  for $Z=\frac{X-11.13}{3.1450}$, the required probability,
  \begin{align*}
    P(X\ge{10})\\
    &= P(\frac{X-11.13}{3.1450}\ge{\frac{10-11.13}{3.1450}})\\
    &= P(Z\ge\frac{-1.13}{3.1450}=-0.3593)\\
    &= 1-P(Z<-0.3593)\\
    &= 1-0.3632\\
    &= \boxed{0.6368}
  \end{align*}

\end{solution}
\end{document}