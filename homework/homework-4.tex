\documentclass[boxes, qed]{homework}
\usepackage{xcolor}
\usepackage{amsmath}
\usepackage{listings}
\name{Rohit Wason}
\course{Math 560}
\term{Spring 2021}
\hwnum{(\#4)}

\newcommand{\bigzero}{\mbox{\normalfont\Large\bfseries 0}}
\newcommand{\rvline}{\hspace*{-\arraycolsep}\vline\hspace*{-\arraycolsep}}

\begin{document}

\newenvironment{amatrix}[1]{%
  \left[\begin{array}{@{}*{#1}{c}|c@{}}
}{%
  \end{array}\right]
}

\newenvironment{augmatrix}[1]{%
  \left[\begin{array}{#1}
}{%
  \end{array}\right]
}

\begin{problem}The distribution of moth counts is discrete and skewed with the mean number of moths
  trapped $\mu=0.5$ and the standard deviation of the number of moths trapped $\sigma=0.8$. A random
  sample of $n=100$ traps will be selected. Use the Central Limit Theorem to approximate the probability that
  the average number of moths in the $100$ traps will be greater than $0.6$.
\end{problem}
\begin{solution}Since the distribution is Normal,
  the mean of the average number of moths, 
  $\mu_{\bar{x}}=\mu=0.5$. Similarly the standard
  deviation of the average number of moths, 
  $\sigma_{\bar{x}}=\frac{\sigma}{\sqrt{n}}=\frac{0.8}{10}=0.08$\\

  Using the Central Limit Theorem, $\bar{x}$ follows the Normal
  distribution, $N(0.5, 0.08)$.\\

  Since the required probability, 
  $P(\bar{x}>0.6)$ is the compliment of the probability
  $P(\bar{x}\le{0.6})$, ``standardizing'' both sized we have that
  $$
  P(\bar{x}\le{0.6}) 
  = P(\frac{\bar{x}-0.5}{0.08}\le{\frac{0.6-0.5}{0.08}})
  = P(z\le{1.25})
  $$
  For $z=\frac{\bar{x}-0.5}{0.08}$. Using the table \textit{Standard Normal Cumulative Probabilities}
  we have $P(z\le{1.25})=0.8944$\\

  Hence the required probability, $P(\bar{x}>0.6)=1-0.8944$
  $$=\boxed{0.1056}$$
\end{solution}

\begin{problem}Consider a population which has a standard deviation $\sigma=10$. 
  A random sample with replacement of size $n$ will be selected and 
  the mean of the sample will be computed. What is the smallest
  sample size $n$ such that the standard deviation of the sampling 
  distribution for the sample mean will be less than $1$?
\end{problem}
\begin{solution}According to CLT 
  the standard deviation of the sample mean, $\sigma_{\bar{x}}$
  is given by $\frac{\sigma}{\sqrt{n}}$.\\

  For $\frac{\sigma}{\sqrt{n}}<1$ we obtain $\sigma^2<n$.
  Or $n>10^2=100$ (since $\sigma=10$).\\

  Hence we see that the smallest sample size for the given condition
  to hold is
  $$\boxed{n=101}$$
\end{solution}

\begin{problem}Suppose that a person’s glucose level one hour after 
  ingesting a sugary drink varies according to a Normal distribution 
  with mean $\mu=130$ mg/dl and standard deviation $\sigma=10$ mg/dl. 
  One hour after injesting a sugary drink, four readings $(n=4)$ are taken of 
  the person’s glucose level and the sample mean of these four readings 
  is computed. (It can be assumed that the readings are independent observations from
  a $N(\mu,\sigma)$ distribution.) Find the value $L$ such that the 
  probability that the sample mean based on the person’s four readings
  falls above $L$ is $0.05$.
\end{problem}
\begin{solution}The mean of the sample mean, $\mu_{\bar{x}}=\mu=130$.
  The standard deviation of the sample mean, 
  $\sigma_{\bar{x}}=\frac{\sigma}{\sqrt{n}}=\frac{10}{2}=5$\\

  The probability that the sample mean, $\bar{x}$ falls above $L$
  can be given by $P(\bar{x}>L)$. On ``standardizing'' the distribution:
  $$P(\frac{\bar{x}-130}{5}>\frac{L-130}{5})$$
  Let's call it $P(z>\frac{L-130}{5})$ which is given to equal $0.05$.
  Using the \textit{Standard Normal Cumulative Probabilities} table
  we can look for $z$ such that $P(z\le\frac{L-130}{5})=1-0.05=0.95$. 
  This gets us $z\approx\frac{1.64+1.65}{2}=1.645$.\\

  Therefore we see that $1.645={\frac{L-130}{5}}$. Or
  $$L={5(1.645)+130=\boxed{138.225}}$$
\end{solution}
\begin{problem}The probability distribution of a random variable
  $X$ is given, and $X_1$ and $X_2$ are random samples of size $2$.
  Assuming $X_1$ and $X_2$ are independent:
\end{problem}
\begin{solution}a) Since the variables are independent,
  the $9$ possible pairs $(X_1,X_2)$ and their probabilities are:
  \begin{enumerate}
    \item $P(0,0)=P(X_1=0 \land X_2=0)=P(X_1=0)P(X_2=0)=0.6^2=0.36$
    \item Similarly $P(0,1)=0.6*0.3=0.18$
    \item $P(0,2)=0.6*0.1=0.06$
    \item $P(1,0)=P(0,1)=0.18$
    \item $P(1,1)=0.3^2=0.09$
    \item $P(1,2)=0.3*0.1=0.03$
    \item $P(2,0)=P(0,2)=0.06$
    \item $P(2,1)=P(1,2)=0.03$, and
    \item $P(2,2)=0.1^2=0.01$\\
  \end{enumerate}

  b) As for the random variable $Y=\frac{X_1+X_2}{2}$
  the different possible values are:\\
  
  \begin{tabular}{|c|c|c|c|}
    \hline
    $Y$ & 0 & 1 & 2 \\
    \hline
    0 & 0 & 0.5 & 1 \\
    \hline
    1 & 0.5 & 1 & 1.5 \\
    \hline
    2 & 1 & 1.5 & 2 \\
    \hline
  \end{tabular}\\

  Therefore the distribution for $Y$ is:\\
  
  \begin{tabular}{|c|c|c|c|c|c|}
    \hline
    $Y$ & 0 & 0.5 & 1 & 1.5 & 2\\
    \hline
    $P(Y)$ & $\frac{1}{9}$ & $\frac{2}{9}$ & $\frac{1}{3}$ & $\frac{2}{9}$ & $\frac{1}{9}$\\
    \hline
  \end{tabular}
\end{solution}
\end{document}