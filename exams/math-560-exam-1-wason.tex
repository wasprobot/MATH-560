\documentclass{article}
\usepackage{amsthm,amssymb,amsmath}
\usepackage[shortlabels]{enumitem}
\usepackage[utf8]{inputenc}
\usepackage[english]{babel}

\author{Rohit Wason}
\newtheorem{problem}{Problem}
\newtheorem*{solution*}{Solution}
\renewcommand{\theenumi}{\alph{enumi}\)}

\title{MATH 560 - Exam 1}
\date{3/3/2021}

\begin{document}
\maketitle
 
\begin{problem}
\end{problem}
\begin{solution*}
  \begin{enumerate}[(a)]
    \item Since the given values of $E$, are exhaustive, 
    the probability that a randomly selected elite typist makes exactly one error in a
    written report
    \begin{align*}
      &=P(E=1) = 1-(P(E=0)+P(E=2))\\
      &=1-(0.63+0.12)\\
      &=\boxed{0.25}
    \end{align*}

    \item The mean of the distribution of $E$ is given by
    \begin{align*}
      \mu_E &= \sum{e_ip_i}\\
      &= 0(0.63)+1(0.25)+2(0.12)\\
      &= \boxed{0.49}
    \end{align*}

    \item For the standard deviation
    \begin{align*}
      \sigma_E^2 &= \sum{e_i^2p_i} - \mu_E^2\\
      &= 0^2(0.63)+1^2(0.25)+2^2(0.12) - 0.49^2\\
      &= \boxed{0.4899}
    \end{align*}
    Therefore $\sigma_E\approx{0.6999}$

    \item To get the probability that at least one of three S.R.S has two errors
    we observe that the probability that a randomly selected person does not have $2$ errors is
    $P(2')=P(0)+P(1)=0.88$.\\

    Since the events of selecting the 3 persons are independent of each other
    (selecting one person doesn’t affect which subsequent person gets picked), the
    required probability is $P(2')^3\approx{\boxed{0.6814}}$
  \end{enumerate}
\end{solution*}

\begin{problem}
\end{problem}
\begin{solution*}It is given that\\
  $\mu_G=134$\\
  $\sigma_G=78$\\
  $\mu_E=188$\\
  $\sigma_E=46$\\
  $\rho_{GE}=-0.5$
  \begin{enumerate}[(a)]
    \item The monthly mean for the total utility bill,
      $$\mu_{G+E}=\mu_G+\mu_E=134+188=322$$
    \item For the monthly standard deviation for the total utility bill,
      $$
        \sigma_{G+E}^2
        =\sigma_G^2+\sigma_E^2+2\rho\sigma_G\sigma_E
        =78^2+46^2+2(-0.5)(78\times 46)
        =4612
      $$
      Therefore $\sigma_{G+E}\approx{67.91}$
  \end{enumerate}
\end{solution*}

\begin{problem}
\end{problem}
\begin{solution*}A skewed continuous distribution is given with\\
  $\mu=40$\\
  $\sigma=40$\\
  $n=64$
  \begin{enumerate}[(a)]
    \item Since the distribution is Normal, the mean 
    of the sample mean is same as population mean. Therefore
      $$\mu_{\bar{x}}=\mu=40$$
    \item Similarly, the standard deviation of the sample mean 
    is given by
      $$\sigma_{\bar{x}}=\sigma/\sqrt{n}=40/\sqrt{64}=5$$
    \item The Central Limit Theorem states that the \textbf{mean time between texts},
    $\bar{x}$ follows the Normal distribution, $N(\mu_{\bar{x}}, \sigma_{\bar{x}})$ and is given by
      $$N(40,5)$$
      and the probability that the mean time for 64 independent pairs of consecutive text messages 
      is more than 42 minutes is
      \begin{align*}
        &P(\bar{x}>42)\\
        &=P(z>\frac{42-40}{40})\\
        &=P(z>0.05) = 1-P(z\le{0.05})\\
        &=1-pnorm(0.05)\approx\boxed{0.48}
      \end{align*}
  \end{enumerate}
\end{solution*}

\begin{problem}
\end{problem}
\begin{solution*}The following are given:\\
  $n=25$\\
  $\sigma=3$\\
  $\mu_{\bar{x}}=5.43$\\
  $\sigma_{\bar{x}}=2.98$
  \begin{enumerate}[(a)]
  \item To compute a $90\%$ two-sided confidence interval,
    we first note that since the Normal Table evaluates probability
    \textbf{up to} a certain statistic, let's call it $z^*$,
    we need to ``add the tail'' to the probability before 
    we look up $z^*$. In other words,
    \begin{equation}
      z^*=qnorm(.90+\frac{1-.90}{2})=qnorm(.95)\approx{1.6449}
    \end{equation}
    The required \textbf{margin of error} is given by
    \begin{equation}
      m=z^*\sigma/\sqrt{n}\approx 0.9869
    \end{equation}
    The $90\%$ two-sided confidence interval, therefore
    \begin{align*}
      CI
      &=(\mu_{\bar{X}}-0.9869, \mu_{\bar{X}}+0.9869)\\
      &=(5.43-0.9869, 5.43+0.9869)\\
      &=\boxed{(4.4431, 6.4169)}
    \end{align*}
    
    \item To find the smallest sample size needed so that the margin of
    error, $m$, is no more than $0.5$ we use (2) above,
    $$m = z^*\sigma/\sqrt{n} = 1.6449(3)/\sqrt{n}$$
    For the desired inequality $m\le{0.5}$, we can say
    \begin{align*}
      \frac{1.6449(3)}{\sqrt{n}} &\le 0.5\\
      \sqrt{n} &\ge 9.87\\
      n &\ge 97.4
    \end{align*}
    We conclude that the sample size must be \textbf{at least 98}.
\end{enumerate}

\end{solution*}

\begin{problem}
\end{problem}
\begin{solution*}The following are given:\\
    $\mu_0=8$\\
    $n=6$\\
    $\bar{x} = mean(c(7.7, 7.4, 7.5, 7.9, 6.8, 7.9)) \approx 7.53$\\
    $\sigma=0.5$\\
    $\alpha=0.05$\\

    The \textbf{null-hypothesis} is stated as $H_0:\mu=8$ and the \textbf{alternate hypothesis} is
    $H_a:\mu\ne{8}$.\\

    The \textbf{test statistic} we want to calculate is:
    \begin{align*}
        z&=\frac{\bar{x}-\mu_0}{\sigma/\sqrt{n}}\\
        &=\frac{7.53-8}{0.5/\sqrt{6}}\\
        &\approx -2.3025
    \end{align*}
    Looking up $-2.3025$ in the Normal table, we get $P(Z<z)=0.01065292$
    however since this is a \textbf{two-sided} alternate (less than or greater than $\mu_0$),
    we shall use $2\times P(Z<z)=0.02130584$.\\

    This value is significantly smaller than significance-level, $\alpha=.05$.
    Hence $H_0$ is to be \textbf{rejected}.
\end{solution*}

\end{document}
